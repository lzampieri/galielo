\documentclass[11pt,a4 paper]{article}
\usepackage{amsmath, amsthm} 
\usepackage[italian]{babel}
\usepackage[T1]{fontenc}
\usepackage[utf8]{inputenc}
\usepackage{graphicx}
\usepackage{caption}
\usepackage{siunitx}
\captionsetup{tableposition=top,font=small}
\usepackage{booktabs}
\usepackage{tabularx}
\usepackage{gensymb}
\usepackage{tablefootnote}
\usepackage{textcomp} 
\usepackage{enumitem}
\setlist[description]{font={\scshape}}
\usepackage{wrapfig}
\usepackage{float}
\usepackage{floatflt}
\usepackage{commath}
\usepackage{circuitikz}
\usepackage{bm}


\begin{document}
	
	\section*{Metodo di assegnazione dei punti}
	
		
	Siano $a_1$, $d_1$, $a_2$ e $d_2$ gli attuali punteggi di attaccanti e difensori rispettivamente della squadra vincente e di quella perdente, e siano $p_1$ e $p_2$ i punteggi delle due squadre; ovviamente, $p_1=10$.
	
	Per le partite che finiscono ai vantaggi si consideri un punteggio di 10-9). 
	
	I punti guadagnati sono calcolati dal prodotto di tre fattori:
	
	\begin{itemize}
		\item Il fattore di \emph{risultato}, dato dalla formula:
		\[
		f_r = 34e^{-0.14p_2}
		\]
		\item Il fattore di \emph{potenza} delle squadre, calcolato come:
		\begin{gather*}
		s_1 = \sqrt[3]{\frac{a_1^3+d_1^3}{2}} \\
		s_2 = \sqrt[3]{\frac{a_2^3+d_2^3}{2}} \\
		f_p = 1 - \frac{1}{1+10^{\frac{s_2-s_1}{400}}}
		\end{gather*}
		\item Il fattore di \emph{merito}, diverso per ogni giocatore:
		\begin{gather*}
		f_{m,a_1} = \frac{2d_1}{a_1+d_1} \qquad
		f_{m,d_1} = \frac{2a_1}{a_1+d_1} \\
		f_{m,a_2} = \frac{2a_2}{a_2+d_2} \qquad
		f_{m,d_2} = \frac{2d_2}{a_2+d_2}
		\end{gather*}
		
	\end{itemize}

	I giocatori della prima squadra guadagneranno quindi $f_p\cdot f_r\cdot f_m$ punti, mentre quelli della seconda ne perderanno $f_p\cdot f_r\cdot f_m$, arrotondando i risultati per difetto.
	
	
	
	\rightline{-Krökel}
\end{document}
